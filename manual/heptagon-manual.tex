\documentclass[a4paper]{article}

\usepackage[T1]{fontenc}
\usepackage[utf8]{inputenc}
\usepackage[a4paper]{geometry}
%\usepackage[francais]{babel}
%\usepackage{subfigure}
%\usepackage{fancyvrb}
%\usepackage{fancyhdr}
\usepackage[hypertex,ps2pdf]{hyperref}
\usepackage{array}
\usepackage{xcolor}
%\usepackage{comment}
%\usepackage{lmodern}
\usepackage{varwidth}
%\usepackage{tikz}
%\usetikzlibrary{arrows}
%\usetikzlibrary{automata}
%\usetikzlibrary{matrix}
%\usetikzlibrary{shapes}
%\usetikzlibrary{positioning}
\usepackage{macros}

% fontes tt avec gras (mots-clés)
\renewcommand{\ttdefault}{txtt}

\lstset{
  language=Heptagon,% numbers=left, numberstyle=\small,
  basicstyle=\normalsize\ttfamily,captionpos=b,
  frame={tb}, rulesep=1pt, columns=fullflexible,
  xleftmargin=1cm, xrightmargin=1cm,
  mathescape=true
}

\title{Heptagon/BZR manual}

\author{}

%\date{}

\begin{document}

\maketitle

\section{Introduction and tutorial}
\label{sec:intro}

\subsection{Heptagon: short presentation}
\label{sec:hept-short-pres}

Heptagon is a synchronous dataflow language, with a syntax allowing the
expression of control structures (e.g., switch or mode automata).

A typical Heptagon program will take as input a sequence of values, and will
output a sequence of values. Then, variables (inputs, outputs or locals) as well
as constants are actually variable or constant \emph{streams}. The usual
operators (e.g., arithmetic or Boolean operators) are applied pointwise on these
sequences of values.

For example, the Heptagon program below is composed of one node \texttt{plus},
performing the pointwise sum of its two integer inputs:

\begin{lstlisting}
node plus(x:int,y:int) returns (z:int)
let
  z = x + y;
tel
\end{lstlisting}

\texttt{x} and \texttt{y} are the inputs of the node \texttt{plus}; \texttt{z}
is the output. \texttt{x}, \texttt{y} and \texttt{z} are of type \texttt{int},
denoting integer \emph{streams}. \texttt{z} is defined by the equation
\lstinline|z = x + y|.

An execution of the node \texttt{plus} can then be:
\[
\begin{streams}{5}
  x & 1 & 2 & 3 & 4 & \ldots\\\hline
  y & 1 & 2 & 1 & 2 & \ldots\\\hline
  \mathtt{plus}(x,y) & 2 & 4 & 4 & 6 & \ldots\\
\end{streams}
\]

\subsection{Compilation}
\label{sec:compilation}

The Heptagon compiler is named \texttt{heptc}. Its list of options is available by
:

\begin{alltt}
> heptc -help
\end{alltt}

Every options described below are cumulable.

Assuming that the program to compile is in a file named \texttt{example.ept},
then one can compile it by typing :

\begin{alltt}
> heptc example.ept
\end{alltt}

However, such compilation will only perform standard analysis (such as typing,
causality, scheduling) and output intermediate object code, but not any final or
executable code.

The Heptagon compiler can thus generate code in some general languages, in order
to obtain either a standalone executable, or a linkable library. The target
language must then be given by the \texttt{-target} option:

\begin{alltt}
> heptc -target <language> example.ept
\end{alltt}

Where \texttt{<language>} is the name of the target language. For now, available
languages are C (\texttt{c} option) and Java (\texttt{java} option).

\subsection{Generated code}
\label{sec:generated-code}

The generic generated code consists, for each node, of two imperative functions:
\begin{itemize}
\item one ``reset'' function, used to reset the internal memory of the node;
\item one ``step'' function, taking as input the nodes inputs, and whose call
  performs one step of the node, updates the memory, and outputs the nodes
  outputs.
\end{itemize}

A standard way to execute Heptagon program is to compile the generated files
together with a main program of the following scheme :

\begin{alltt}
call the \textit{reset} function
for each instant
   get the \textit{inputs} values
   \textit{outputs} \(\leftarrow\) \textit{step(inputs)}
   do something with \textit{outputs} values
\end{alltt}

Appendix~\ref{sec:app-generated-code} give specific technical details for each target language.


\subsection{Simulation}
\label{sec:simulation}

A graphical simulator is available: \texttt{hepts}. It allows the user to simulate
one node by providing a graphical window, where simulation steps can be
performed by providing inputs of the simulated node.

This simulator tool interacts with an executable, typically issued of Heptagon
programs compilation, and which await on the standard input the list of the
simulated node's inputs, and prints its outputs on the standard output. Such
executable, for the simulation of the node \texttt{f}, can be obtained by the
\texttt{-s <node>} option:
\begin{alltt}
> heptc -target c -s f example.ept
\end{alltt}

We can then directly compile the generated C program (whose main function stand
in the \texttt{\_main.c} file):
\begin{alltt}
> cd example_c
> gcc -Wall -c example.c
> gcc -Wall -c _main.c
> gcc -o f_sim _main.o example.o       # \text{executable creation}
\end{alltt}

This executable \texttt{f\_sim} can then be used with the graphical simulator
\texttt{hepts}, which takes as argument:
\begin{itemize}
\item The name of the module (capitalized name of the program without the
  \texttt{.ept} extension),
\item the name of the simulated node,
\item the path to the executable \texttt{f\_sim}.
\end{itemize}
\begin{alltt}
> hepts -mod Example -node f -exec example_c/f_sim
\end{alltt}

\section{Syntax and informal semantics}
\label{sec:synt-infor-sem}

Heptagon programs are synchronous Moore machines, with parallel and hierarchical
composition. The states of such machines define dataflow equations. The
Figure~\ref{fig:mixed-state-dataflow-example} gives an example of such program.

\begin{figure}[htbp]
  \centering
  \includegraphics{figures/mixed-state-df}
  \caption{Mixed state and dataflow example}
  \label{fig:mixed-state-dataflow-example}
\end{figure}

\subsection{Nodes}
\label{sec:nodes}

Heptagon programs are structured in \emph{nodes}: a program is a sequence of
nodes. A node is a subprogram with a name $f$, inputs $\ton{x}{,}$, outputs
$\ton[1][p]{y}{,}$, local variables $\ton[1][q]{z}{,}$ and declarations
$D$. $y_i$ and $z_i$ variables are to be defined in $D$, using operations
between values of $x_j$, $y_j$, $z_j$. Figure~\ref{fig:syntax-nodes} gives the
syntax of node definitions, together with a graphical syntax used in this
manuel\footnote{declaration of local variables are mandatory for the compiler in
  the textual syntax, however we will sometimes omit it in the graphical syntax
  for the sake of brevity}. The declaration of one variable comes with its type
($t_i$, $t'_i$ and $t''_i$ being the type of respectively $x_i$, $y_i$ and
$z_i$).

\begin{figure}[htb]
  \centering
%  \begin{varwidth}{\linewidth}
    \[
    \begin{array}{|c|c}
      \cline{1-1}
      f(x_1:t_1,\ldots,x_n:t_n) = y_1:t'_1,\ldots,y_p:t'_p & \\\hline
      \multicolumn{2}{|c|}{}\\
      \multicolumn{2}{|c|}{D}\\
      \multicolumn{2}{|c|}{}\\\hline
    \end{array}
    \]
%  \end{varwidth}\hspace{1cm}
%  \begin{varwidth}{\linewidth}
\begin{lstlisting}
node f(x$_1$:t$_1$;$\ldots$;x$_n$:t$_n$) returns (y$_1$:t$'_1$,$\ldots$,y$_p$:t$'_p$)
  var z$_1$:t$''_1$,$\ldots$,z$_q$:t$''_q$;
let
  D
tel
\end{lstlisting}
%    \end{varwidth}
  \caption{Graphical and textual syntax of node definition}
  \label{fig:syntax-nodes}
\end{figure}

The program of the Figure~\ref{fig:mixed-state-dataflow-example} can thus be
structured as the semantically equivalent program of the
Figure~\ref{fig:struct-prog-example}. The Figure~\ref{fig:textual-syntax} gives
the textual syntax of this program.


\begin{figure}[htbp]
  \centering
  \includegraphics{figures/struct-pg}
  \caption{Structured program example}
  \label{fig:struct-prog-example}
\end{figure}

\begin{figure}[htbp]
  \centering

\begin{lstlisting}
node h(a:bool) returns (y:bool)
  let
    automaton
      state Idle
        do y = false
        until a then Active
      state Active
        do y = true
        until a then Idle
    end
  tel

node g (a,b:bool) returns (y:bool)
  var y1,y2 : bool;
  let
    y = y1 & y2;
    y1 = h(a);
    y2 = h(b);
  tel

node f (c,d:bool) returns (y:bool)
  let
    automaton
      state A
        do y = false
        until c then B
      state B
        do y = g(c,d)
        until c & d then C
      state C
        do y = true
        until d then A
    end
  tel
\end{lstlisting}

  \caption{Textual syntax}
  \label{fig:textual-syntax}
\end{figure}


Heptagon allows to distinguish, by mean of clocks and control structures (switch,
automata), for declarations and expressions, the discrete instants of
activation, when the declarations and expressions are computed and progress
toward further states, and other instants when neither computation nor
progression are performed.

\subsection{Expressions}
\label{sec:expressions}

\subsubsection{Values and combinatorial operations}
\label{sec:variables-constants}

Heptagon is a dataflow language, i.e., every value, variable or constant, is
actually a stream of value. The usual operators (e.g., arithmetic or Boolean
operators) are applied pointwise on these sequences of values, as combinatorial
operations (as opposed to \emph{sequential} operations, taking into account the
current \emph{state} of the program: see delays in Section~\ref{sec:delays}).

Thus, \texttt{x} denotes the stream $x_1.x_2.\ldots$, and \lstinline|x + y| is
the stream defined by $($\lstinline|x + y|$)_i=x_i+y_i$.

\[
\begin{streams}{5}
  \mathtt{x} & x_1 & x_2 & x_3 & x_4 & \ldots\\\hline
  \mathtt{y} & y_1 & y_2 & y_3 & y_4 & \ldots\\\hline
  \mathtt{x + y} & x_1+y_1 & x_2+y_2 & x_3+y_3 & x_4+y_4 & \ldots\\
\end{streams}
\]


\subsubsection{Delays}
\label{sec:delays}

Delays are the way to introduce some state in a Heptagon program.

\begin{itemize}
\item \lstinline|pre x| gives the value of \texttt{x} at the preceding
  instant. The value at the first instant is undefined.
\item \lstinline|x -> y| takes the value of \texttt{x} at the first instant,
  and then the value of \texttt{y};
\item \lstinline|x fby y| is equivalent to \lstinline|x -> pre y|.
\end{itemize}

\[
\begin{streams}{3}
  \text{\lstinline|x|} & x_1 & x_2 & x_3 \\
  \hline
  \text{\lstinline|y|} & y_1 & y_2 & y_3 \\
  \hline
  \text{\lstinline|pre x|} & \perp & x_1 & x_2 \\
  \hline
  \text{\lstinline|x -> y|} & x_1 & y_2 & y_3 \\
  \hline
  \text{\lstinline|x fby y|} & x_1 & y_1 & y_2 \\
\end{streams}
\]


\subsection{Declarations}
\label{sec:declarations}

A declaration $D$ can be either :
\begin{itemize}
\item an equation $x = e$, defining variable $x$ by the expression $e$ at each
  activation instants ;
\item a node application $(\tonp{y}{,}) = f(\ton{e}{,})$, defining variables
  $\tonp{y}{,}$ by application of the node $f$ with values $\ton{e}{,}$ at each
  activation instants ;
\item parallel declarations of $D_1$ and $D_2$, noted graphically $D_1\vdots
  D_2$ and textually $D_1\Pv D_2$. Variables defined in $D_1$ and $D_2$ must be
  exclusive. The activation of this parallel declaration activate both $D_1$ and
  $D_2$, which are both computed and both progress ;
\item a switch control structure ;
\item an automaton.
\end{itemize}

\subsubsection{Switch control structures}
\label{sec:switch-contr-struct}

The \texttt{switch} control structure allows to controls which equations are
evaluated:

\begin{lstlisting}
type modes = Up | Down

node two(m:modes;v:int) returns (o:int)
  var last x:int = 0;
let
  o = x;
  switch m
  | Up do x = last x + v
  | Down do x = last x - v
  end
tel
\end{lstlisting}

The \texttt{last} keyword defines a memory which is shared by the different
modes. Thus, \lstinline|last x| is the value of the variable \texttt{x} in the
previous instant, whichever was the activated mode.

\subsubsection{Automata}
\label{sec:automata}

An automaton is a set of states (one of which being the initial one), and
transitions between these states, triggered by Boolean expressions. A
declaration is associated to each state. The set of variables defined by the
automaton is the union, not necessarily disjoint (variables can have different
definitions in different states, and can be partially defined : in this case,
when the variable is not defined in an active state, the previous value of this
variable is taken.

At each automaton activation instant, one and only one state of this automaton
is active (the initial one at the first activation instant). The declaration
associated to this active state is itself activated and progress in this
activation instant.

\paragraph{Example}
\label{sec:example}

The following example gives the node \texttt{updown}. This node is defined by an
automaton composed of two states:
\begin{itemize}
\item the state \texttt{Up} gives to \texttt{x} its previous value augmented of 1
\item the state \texttt{Down} gives to \texttt{x} its previous value diminued of 1
\end{itemize}
This automaton comprises two transitions:
\begin{itemize}
\item it goes from \texttt{Up} (the initial state) to \texttt{Down} when
  \texttt{x} becomes greater or equal than 10;
\item it goes from \texttt{Down} to \texttt{Up} when \texttt{x} becomes less or
  equal 0.
\end{itemize}

\begin{lstlisting}
node updown() returns (y:int)
  var last x:int = 0;
let
  y = x;
  automaton
    state Up
      do x = last x + 1
      until x >= 10 then Down
    state Down
      do x = last x - 1
      until x <= 0 then Up
  end
tel
\end{lstlisting}

\[
\begin{streams}{14}
\text{current state} & Up & Up & Up & Up & Up & Up & Up & Up & Up & Up & Down & Down & Down & \ldots\\\hline
\mathtt{y} & 1 & 2 & 3 & 4 & 5 & 6 & 7 & 8 & 9 & 10 & 9 & 8 & 7 & \ldots\\\hline
\end{streams}
\]

 Expressions on outgoing transitions of this active state are
evaluated, so as to compute the next active state : these are weak
transitions. Transitions are evaluated in declaration order, in the textual
syntax. If no transition can be triggered, then the current state is the next
active state.




\section{BZR: Contracts for controller synthesis}
\label{sec:extens-with-contr}

Contracts are an extension of the Heptagon language, so as to allow to perform
discrete controller synthesis on Heptagon programs. The extended language is
named BZR.

We associate to each node a \emph{contract}, which is a program associated with
two outputs :
\begin{itemize}
\item an output $e_A$ representing the environment model ;
\item an invariance objective $e_G$ ;
\item a set $\set{\ton{c}{,}}$ of controllable variables used for ensuring this objective.
\end{itemize}

This contract means that the node will be controlled, i.e., that values will be
given to $\ton{c}{,}$ such that, given any input trace yielding $e_A$, the
output trace will yield the true value for $e_G$.

\begin{center}
\includegraphics{figures/node-contract}
\end{center}

In the textual syntax, the contracts are noted :
\begin{lstlisting}
node f(x$_1$:t$_1$;$\ldots$;x$_n$:t$_n$) returns (y$_1$:t$'_1$;$\ldots$;y$_p$:t$'_p$)
contract
  var $\ldots$
  let
     $\ldots$
  tel
  assume $e_A$
  enforce $e_G$
  with (c$_1$:t$''_1$;$\ldots$;c$_q$:t$''_n$)

var $\ldots$
let
  y$_1$ = f$_1$($\ton{\mathtt{x}}{,},\ton[1][q]{\mathtt{c}}{,}$); 
  $\vdots$
  y$_p$ = f$_p$($\ton{\mathtt{x}}{,},\ton[1][q]{\mathtt{c}}{,}$); 
tel
\end{lstlisting}

\section{BZR Running Example:  Multi-task System}
\label{sec:multi-task-system}

\subsection{Delayable Tasks}
\label{sec:delayable-tasks}



We consider a multi-task system composed of $n$ delayable
tasks. Figure~\ref{fig:del-task} shows a delayable task. A delayable task takes
three inputs \texttt{r}, \texttt{c} and \texttt{e}: \texttt{r} is the task
launch request from the environment, \texttt{e} is the end request, and
\texttt{c} is meant to be a controllable input controlling whether, on request,
the task is actually launched (and therefore goes in the active state), or
delayed (and then forced by the controller to go in the waiting state by stating
the false value to \texttt{c}). This node outputs a unique boolean \texttt{act}
which is true when the task is in the active state.

\begin{figure}[htb]
\begin{lstlisting}
node delayable(r,c,e:bool) returns (act:bool)
  let
    automaton
      state Idle
        do act = false
        until r & c then Active
            | a & not c then Wait
      state Wait
        do act = false
        until c then Active
      state Active
        do act = true
        until e then Idle
    end
  tel
\end{lstlisting}
\caption{Delayable task}
\label{fig:del-task}
\end{figure}

The Figure~\ref{fig:n-del-task} shows then a node \texttt{ntasks} where $n$
delayable tasks have been put in parallel. The tasks are inlined so as to be
able to perform DSC on this node, taking into account the tasks' states. Until
now, the only interest of modularity is, from the programmer's point of view, to
be able to give once the delayable task code.

\begin{figure}[htb]
\begin{lstlisting}
node ntasks($\ton{\mathtt{r}}{,},\ton{\mathtt{e}}{,}$:bool)
       returns ($\ton{\mathtt{a}}{,}$:bool)
  contract
  let
    ca$_{1}$ = a$_{1}$ & (a$_{2}$ or $\ldots$ or a$_{n}$);
    $\vdots$
    ca$_{n-1}$ = a$_{n-1}$ & a$_{n}$;
  tel
  enforce not (ca$_{1}$ or \ldots or ca$_{n-1}$) 
  with ($\ton{\mathtt{c}}{,}$:bool)
let
  a$_{1}$ = inlined delayable(r$_{1}$,c$_{1}$,e$_{1}$); 
  $\vdots$
  a$_{n}$ = inlined delayable(r$_{n}$,c$_{n}$,e$_{n}$); 
tel
\end{lstlisting}
\caption{\texttt{ntasks} node: $n$ delayable tasks in parallel}
\label{fig:n-del-task}
\end{figure}

This \texttt{ntasks} node is provided with a contract, stating that its
composing tasks are exclusive, i.e., that there are no two tasks in the active
state at the same instant. This contract is enforced with the help of the
controllable inputs $c_i$.

\subsection{Contract composition}
\label{sec:contract-composition}

We want know to reuse the \texttt{ntasks} node, in order to build modularly a
system composed of $2n$ tasks. The Figure~\ref{fig:2n-del-task} shows the
parallel composition of two \texttt{ntasks} nodes. We associate to this
composition a new contract, which role is to enforce the exclusivity of the $2n$
tasks.

\begin{figure}[htb]
\begin{lstlisting}
node main($\ton[1][2n]{\mathtt{r}}{,},\ton[1][2n]{\mathtt{e}}{,}$:bool)
       returns ($\ton[1][2n]{\mathtt{a}}{,}$:bool)
  contract
  let
    ca$_{1}$ = a$_{1}$ & (a$_{2}$ or $\ldots$ or a$_{2n}$);
    $\vdots$
    ca$_{2n-1}$ = a$_{2n-1}$ & a$_{2n}$;
  tel
  enforce not (ca$_{1}$ or $\ldots$ or ca$_{2n-1}$)
let
  ($\ton{\mathtt{a}}{,}$) = ntasks($\ton{\mathtt{r}}{,}$,$\ton{\mathtt{e}}{,}$); 
  ($\ton[n+1][2n]{\mathtt{a}}{,}$) = ntasks($\ton[n+1][2n]{\mathtt{r}}{,}$,$\ton[n+1][2n]{\mathtt{e}}{,}$); 
tel
\end{lstlisting}
\caption{Composition of two \texttt{ntasks} nodes}
\label{fig:2n-del-task}
\end{figure}

It is easy to see that the contract of \texttt{ntasks} is not precise enough to
be able to compose several of these nodes. Therefore, we need to refine this
contract by adding some way to externally control the activity of the tasks.

\subsection{Contract refinement}
\label{sec:contract-refinement}

We first add an input \texttt{c}, meant to be controllable. The refined contract
will enforce that:
\begin{enumerate}
\item the tasks are exclusive,
\item one task is active only at instants when the input \texttt{c} is
  true. This property, appearing in the contract, allow a node instantiating
  \texttt{ntasks} to forbid any activity of the $n$ tasks instantiated.
\end{enumerate}
The Figure~\ref{fig:n-del-task-2} contains this new \texttt{ntasks} node.

\begin{figure}[htb]
\begin{lstlisting}
node ntasks(c,$\ton{\mathtt{r}}{,}$,$\ton{\mathtt{e}}{,}$:bool) returns ($\ton{\mathtt{a}}{,}$:bool)
  contract
  let
    ca$_{1}$ = a$_{1}$ & (a$_{2}$ or $\ldots$ or a$_{n}$);$\ldots$
    ca$_{n-1}$ = a$_{n-1}$ & a$_{n}$;
    one = a$_{1}$ or $\ldots$ or a$_{n}$;
  tel
  enforce not (ca$_{1}$ or $\ldots$ or ca$_{n-1}$) & (c or not one)
  with ($\ton{\mathtt{c}}{,}$:bool)
let
  a$_{1}$ = inlined delayable(r$_{1}$,c$_{1}$,e$_{1}$); 
  $\vdots$
  a$_{n}$ = inlined delayable(r$_{n}$,c$_{n}$,e$_{n}$); 
tel
\end{lstlisting}
\caption{First contract refinement for the \texttt{ntasks} node}
\label{fig:n-del-task-2}
\end{figure}

However, the controllability introduced here is know too strong. The synthesis
will succeed, but the computed controller, without knowing how \texttt{c} will
be instantiated, will actually block every tasks in their idle state. Indeed, if
the controller allows one task to go in its active state, the input \texttt{c}
can become false at the next instant, violating the property to enforce.

Thus, we propose to add an assumption to this contract: the input \texttt{c}
will not become false if a task was active an instant before. This new contract
is visible in Figure~\ref{fig:n-del-tasks-3}.

\begin{figure}[htb]
  \centering
\begin{lstlisting}
node ntasks(c,$\ton{\mathtt{r}}{,}$,$\ton{\mathtt{e}}{,}$:bool) returns ($\ton{\mathtt{a}}{,}$:bool)
  contract
  let
    ca$_{1}$ = a$_{1}$ & (a$_{2}$ or $\ldots$ or a$_{n}$);$\ldots$
    ca$_{n-1}$ = a$_{n-1}$ & a$_{n}$;
    one =  a$_{1}$ or $\ldots$ or a$_{n}$;
    pone = false fby one;
  tel
  assume (not pone or c)
  enforce not (ca$_{1}$ or $\ldots$ or ca$_{n-1}$) & (c or not one)
  with ($\ton{\mathtt{c}}{,}$)
let
  a$_{1}$ = inlined delayable(r$_{1}$,c$_{1}$,e$_{1}$); 
  $\vdots$
  a$_{n}$ = inlined delayable(r$_{n}$,c$_{n}$,e$_{n}$); 
tel
\end{lstlisting}
  \caption{Second contract refinement for the \texttt{ntasks} node}
  \label{fig:n-del-tasks-3}
\end{figure}

We can then use this new \texttt{ntasks} version for the parallel composition,
by instantiating the \texttt{c} input by a controllable variable and its
negation. This composition can be found in Figure~\ref{fig:ntasks-compos}.

\begin{figure}[htb]
  \centering
\begin{lstlisting}
node main($\ton[1][2n]{\mathtt{r}}{,}$,$\ton[1][2n]{\mathtt{e}}{,}$:bool) returns ($\ton[1][2n]{\mathtt{a}}{,}$:bool)
  contract
  let
    ca$_{1}$ = a$_{1}$ & (a$_{2}$ or $\ldots$ or a$_{2n}$);
    $\vdots$
    ca$_{2n-1}$ = a$_{2n-1}$ & a$_{2n}$;
  tel
  enforce not (ca$_{1}$ or $\ldots$ or ca$_{2n-1}$)
  with (c:bool)
let
  ($\ton{\mathtt{a}}{,}$) = ntasks(c,$\ton{\mathtt{r}}{,}$,$\ton{\mathtt{e}}{,}$); 
  ($\ton[n+1][2n]{\mathtt{a}}{,}$) = ntasks(\Not c,$\ton[n+1][2n]{\mathtt{r}}{,}$,$\ton[n+1][2n]{\mathtt{e}}{,}$); 
tel
\end{lstlisting}
  \caption{Two \texttt{ntasks} parallel composition}
  \label{fig:ntasks-compos}
\end{figure}


\appendix

\section{Generated code}
\label{sec:app-generated-code}

\subsection{C generated code}
\label{sec:c-generated-code}

C generated files from an Heptagon program \texttt{example.ept} are placed in a
directory named \texttt{example\_c}. This directory contains one file
\texttt{example.c}. For each node \texttt{f} of the source program, assuming
that \texttt{f} has inputs $(x_1:t_1,\ldots,x_n:t_n)$ and outputs
$(y_1:t'_1,\ldots,y_p:t'_p)$, $t_i$ and $t'_i$ being the data types of these
inputs and outputs, then the \texttt{example.c} file contains, for each node
\texttt{f}:

\begin{itemize}
\item A \texttt{Example\_\_f\_reset} function, with an argument \texttt{self} being a
  memory structure instance:

\begin{lstlisting}[language=C]
void Example__f_reset(Example__f_mem* self);
\end{lstlisting}

\item A \texttt{Example\_\_f\_step} function, with as arguments the nodes inputs, a
  structure \texttt{\_out} where the output will be put, and a memory structure
  instance \texttt{self}:

\begin{lstlisting}[language=C]
void Example__f_step(t$_{1}$ x$_{1}$, ..., t$_{n}$ x$_{n}$,
                     Example__f_out* \_out,
                     Example__f_mem* self);
\end{lstlisting}

After the call of this function, the structure \texttt{\_out} contains the
outputs of the node:
\begin{lstlisting}[language=C]
typedef struct \{
  t$'_1$ y$_{1}$;
  ...
  t$'_p$ y$_{p}$;
\} Example__f_ans;
\end{lstlisting}
\end{itemize}

An example of main C code for the execution of this node would be then:
\begin{lstlisting}[language=C]
#include "example.h"

int main(int argc, char * argv[]) \{
  
  Example__f_m mem;
  t$_{1}$ x$_{1}$;
  ...
  t$_{n}$ x$_{n}$;
  Example__f_out ans;

  /* initialize memory instance */
  f_reset(&mem);
  
  while(1) \{
    /* read inputs */
    scanf("...", &x$_{1}$, ..., &x$_{n}$);

    /* perform step */
    Example__f_step(x$_{1}$, ..., x$_{n}$, &ans, &mem);

    /* write outputs */
    printf("...", ans.y$_{1}$, ..., ans.y$_{p}$);
  \}
\}
\end{lstlisting}

The above code is nearly what is produce for the simulator with the \texttt{-s}
option (see Section~\ref{sec:simulation}).

% \subsection{OCaml generated code}
% \label{sec:ocaml-generated-code}


% If the option \texttt{-target caml} is given, then the compiler generates OCaml
% code in a file named \texttt{example.ml}. Heptagon nodes are compiled into OCaml
% classes, where state variables are class properties, and the two functions
% ``reset'' and ``step'' are class methods. Thus, the class type of \texttt{f}
% would be:
% \begin{alltt}
% class f :
%   object
%     method reset : unit \(\rightarrow\) unit
%     method step : t\ind{1} * ... * t\ind{n} \(\rightarrow\) (t\('\sb{1}\) * ... * t\('\sb{p}\))
%   end
% \end{alltt}

\subsection{Java generated code}
\label{sec:java-generated-code}

Java generated files from an Heptagon program \texttt{example.ept} are placed in
a directory named \texttt{example\_java}. This directory contains one Java class
\texttt{f} (in the file \texttt{f.java}) for each node \texttt{f} of the source
program. Assuming that \texttt{f} has inputs $(x_1:t_1,\ldots,x_n:t_n)$ and
outputs $(y_1:t'_1,\ldots,y_p:t'_p)$, $t_i$ and $t'_i$ being the data types of
these inputs and outputs, then this \texttt{f} class implements the following
interface:


\begin{lstlisting}[language=Java]
public interface f {
    
    public void reset();

    public fAnswer step(t$_{1}$ x$_{1}$, ..., t$_{n}$ x$_{n}$);
}
\end{lstlisting}

The \texttt{fAnswer} class being a structure containing the outputs:

\begin{lstlisting}[language=Java]
public class fAnswer {
  t$'\sb{1}$ y$_{1}$;
  ...
  t$'\sb{p}$ y$_{p}$;
}
\end{lstlisting}



\end{document}
